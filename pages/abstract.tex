\chapter{\abstractname}

This thesis introduces a novel relation extraction method for deriving the subcellular localization of proteins from the biomedical literature. The task is to extract the semantic relationship between a protein entity and a subcellular location entity as present in the text. To the best of my knowledge, my method is the first ever published to be specifically built for this task. Furthermore, it is the first method to study the extraction of localization relationships from full-text articles.

To train the method, a new corpus "LocText", consisting of 100 MEDLINE abstracts and 4 PMC full-text articles, was annotated. It was found out that, in abstracts, for 81\% of the protein-subcellular location relations, the participating protein and subcellular location entities are present either in the same sentence or in neighboring sentences (1 sentence apart). For the full-text subcorpus, it was found out that 92.2\% of protein-subcellular location relations are same-sentence relations.

Based on these observations, I trained two machine learning models, namely, \textit{SSModel} for extracting same-sentence relations and \textit{DSModel} for extracting different-sentence relations. Both models are \textit{support vector machines} and make heavy use of dependency relations between the tokens and graph-based features. For the abstracts in the LocText corpus, through combined predictions of \textit{SSModel} and \textit{DSModel}, my method achieved a precision of 61.01\%, recall of 66.55\% and F-score of 63.15\%. Additionally, I also present primary evaluations of the \textit{SSModel} on the full-text subcorpus.