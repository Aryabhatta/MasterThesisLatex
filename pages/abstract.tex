\chapter{\abstractname}

%TODO: Abstract
%TODO change for statistics of full text

This thesis introduces a method to extract the instances of subcellular localization of proteins, from the biomedical literature. The extraction of such instances involves extracting the semantic relationship between a protein entity and a subcellular location entity, present in the text. For the purpose of training the method, a new corpus "LocText", consisting of 100 MEDLINE abstracts and 4 PMC full-text articles, is annotated. We found out that for around 77\% of the protein-subcellular location relations in the abstract, the participating protein and subcellular location entities are either present in the same sentence or in neighboring sentences. We try to extract such instances of relations where the participating entities are either in the same sentence or in neighboring sentences.

We train two machine learning models, viz. \textit{SSModel} for extracting same-sentence relations and \textit{DSModel} for extracting different-sentence relations. We extract a lot of graph-based features and make heavy use of dependency relations, while training the models. For the abstracts in the LocText corpus, through combined predictions of \textit{SSModel} and \textit{DSModel}, we could achieve a precision of 66.57\%, recall of 67.61\% and F-score of 66.73\% on the development set and a precision of 61.01\%, recall of 66.55\% and F-score of 63.15\% on the test set.

For full-text subcorpus, we found out that 92.2\% of protein-subcellular location relations are same-sentence relations. We present the results of primary evaluation of \textit{SSModel} on the full-text subcorpus.