\chapter{Conclusion}\label{chapter:conclusion}

In this thesis, I developed a novel method to extract the subcellular localization of proteins from the biomedical literature. I used common natural language processing (NLP) techniques and support vector machines.

Given that existing datasets were not appropriate for the task of protein-location relation extraction, I, along with my thesis advisors, Juan and Tatyana, annotated a new corpus, "LocText". LocText consists of 100 MEDLINE \cite{medline} abstracts, containing mentions of the subcellular localization of diverse proteins. In addition to the abstracts, LocText also contains 4 PMC \cite{pmc} full-text articles. The annotation of documents was done with the help of the \textit{tagtog} web interface \cite{cejuela2014tagtog}.  The documents in LocText were annotated for protein entities, location entities, organism entities, protein-organism relations, and protein-location relations.

We found out that around 60\% of the protein-location relations in the abstract subcorpus are same-sentence relations, i.e., both the protein and location entities are present in the same sentence. Interestingly high, for around 81\% of the protein-location relations in the abstract subcorpus, the participating entities are either present in the same sentence or in neighboring sentences (1 sentence apart).

I developed two models (both SVMs), one for extracting same-sentence relations, \textit{SSModel}, and other one for extracting different-sentence relations, \textit{DSModel}. Despite the name, the \textit{DSModel} was used for extracting relations in which the participating entities are only 1 sentence apart. The feature set of \textit{SSModel} differs only slightly compared to the feature set of the \textit{DSModel}. Features were derived from three sources: text, syntactic parse tree and dependency parse tree. In both models, a sentence is represented as a graph, with the tokens in the sentence forming the nodes of the graph and the dependency relations between the tokens, acting as edges of the graph. Many graph-based features played an important role in prediction performance. Furthermore, I experimented extensively with feature selection. Using \emph{information gain}, I could markedly reduce the number of features while increasing the overall performance of both models: for \textit{SSModel} from 154421 features to 66188 and for \textit{DSModel} from 310512 features to 92514. In some occasions, for example deciding not to use \emph{coreference resolution} in features, speed was preferred over performance. Further possible improvements on the method are explained in Chapter \ref{chapter:outlook}.

I used 5-fold cross validation on the abstract subcorpus for evaluation. The combined predictions of \textit{SSModel} and \textit{DSModel} achieved a precision of 66.57\%, recall of 67.61\% and F-score of 66.73\% on the development set and a precision of 61.01\%, recall of 66.55\% and F-score of 63.15\% on the test set.

For full-text subcorpus, it was found out that 92.2\% of the protein-location relations are same-sentence relations. I used \textit{SSModel} for a preliminary evaluation on the full-text subcorpus. Using 4-fold cross validation, it could achieve a precision of 61.9\%, recall of 86.66\% and F-score of 68.2\% on the test set.

Overall, in this thesis, a novel corpus "LocText" was annotated and a new method for extracting protein-location relations was developed. We published the corpus results \cite{goldberg2015linked} at Biomedical Linked Annotation Hackathon (BLAH) 2015 \cite{blah}. For the research community, we made the corpus available for download, in tagtog format at \url{https://www.tagtog.net/-corpora/loctext}, and in PubAnnotation format at \url{http://pubannotation.org/projects/LocText}. The developed method is first of its kind to be used for extracting subcellular localization of proteins from the natural language. In addition, it is the only method that has performed evaluation on the full-text articles.

%\textbf{TODO}: last paragraph telling the relevance of your work: 1) released novel corpus, already published in a paper and put in 2 external archives, tagtog and PubAnnotation, 2) novel method specifically built for protein-location relation extraction, 3) first evaluation results on full-text articles.