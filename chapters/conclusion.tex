\chapter{Conclusion}\label{chapter:conclusion}

In this thesis, we tried to extract the instances of subcellular localization of proteins, from the biomedical literature. We used natural language processing (NLP) techniques to extract such instances.

As existing datasets were not appropriate for the task of protein-location relation extraction, we annotated a new corpus, "LocText". LocText consists of randomly selected 100 MEDLINE \cite{medline} abstracts, containing mentions of the subcellular localization of proteins. In addition to the abstracts, LocText also contains 4 PMC \cite{pmc} full-text articles. The annotation of both abstracts and full-texts was done with the help of \textit{tagtog} web interface \cite{cejuela2014tagtog}.  The documents in LocText were annotated for protein entities, location entities, organism entities, protein-organism relations, and protein-location relations.

It was found out that around 60\% of the protein-location relations in the abstract subcorpus are same-sentence relations, i.e., both protein and location entity in the relation are present in a single sentence. The remaining 40\% of the protein-location relations in the abstract subcorpus are different-sentence relations, i.e., the participating entities are present in different sentences. The interesting finding was that, for around 81\% of the protein-location relations in the abstract subcorpus, the participating entities are either present in the same sentence or neighboring sentences. We used natural language processing to develop the models that could extract such instances of protein-location relations.

We developed two models, one for extracting same-sentence relations called \textit{SSModel} and other one for extracting different-sentence relations called \textit{DSModel}. Although called different-sentence model, the \textit{DSModel} was used for extracting relations in which the participating entities are present in neighboring sentences. The feature set of \textit{SSModel} differ only slightly compared with the feature set of the \textit{DSModel}. The features for both models were extracted from three sources, viz. text, syntactic parse tree and dependency parse tree. In both models, a sentence is represented as a graph, with the tokens in the sentence forming the nodes of the graph and the dependency relations between the tokens, acting as edges of the graph. Many graph based features played an important role in training the models.

We also performed a lot of experimentation for feature selection. Using information gain of the features, we could reduce the number of features for \textit{SSModel}, from 154421 to 66188 and for \textit{DSModel}, from 310512 to 92514.

We used 5-fold cross validation for evaluation on the abstract subcorpus. The combined predictions of \textit{SSModel} and \textit{DSModel} could achieve a precision of 66.57\%, recall of 67.61\% and F-score of 66.73\% on the development set and a precision of 61.01\%, recall of 66.55\% and F-score of 63.15\% on the test set.

For full-text subcorpus, it was found out that 92.2\% of the protein-location relations are same-sentence relations. We used \textit{SSModel} for preliminary evaluation on the full-text subcorpus. Using 4-fold cross validation, it could achieve a precision of 61.9\%, recall of 86.66\% and F-score of 68.2\% on the test set.
