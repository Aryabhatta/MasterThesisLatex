\chapter{Outlook}\label{chapter:outlook}

This section discusses possible directions for future improvements in a protein-location relation extraction method.

\subsection{Full text}

Section \ref{sec:FTPrimaryRes} presents preliminary performance results on full-text articles. However, in order to have robust performance figures, there is a need to annotate, train, and test on more full-text articles. In addition, specific tuning experiments should be carried out to mold both \textit{SSModel} and \textit{DSModel} to full text.

\subsection{Coreference resolution}

Some experimentation with coreferences was carried out as discussed in Subsection \ref{subsubsec:coref}. However, it was not further studied due to the fact that resolving coreferences using the Stanford CoreNLP pipeline \cite{manning2014stanford} is a time-consuming process. Since my method was developed with the objective of extracting protein-location relations on the fly, i.e., fast, I discarded coreference resolution as a source for features. However, there is scope to use fast coreference resolution from other external methods and this warrants further investigation.

\subsection{Different methods like Markov Logic Networks (MLNs)}


Yoshikawa et al. used Markov Logic Networks \cite{yoshikawa2011coreference} for the task of event extraction and achieved somewhat better results than state-of-the-art SVMs in BioNLP'09. Therefore, it would also be interesting to experiment with MLNs for the task of protein-location relation extraction.